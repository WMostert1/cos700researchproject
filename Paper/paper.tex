
\documentclass[journal,comsoc]{IEEEtran}
\usepackage[T1]{fontenc}% optional T1 font encoding


% Some very useful LaTeX packages include:
% (uncomment the ones you want to load)


\usepackage{float}





% *** CITATION PACKAGES ***
%
\usepackage{cite}
% cite.sty was written by Donald Arseneau
% V1.6 and later of IEEEtran pre-defines the format of the cite.sty package
% \cite{} output to follow that of the IEEE. Loading the cite package will
% result in citation numbers being automatically sorted and properly
% "compressed/ranged". e.g., [1], [9], [2], [7], [5], [6] without using
% cite.sty will become [1], [2], [5]--[7], [9] using cite.sty. cite.sty's
% \cite will automatically add leading space, if needed. Use cite.sty's
% noadjust option (cite.sty V3.8 and later) if you want to turn this off
% such as if a citation ever needs to be enclosed in parenthesis.
% cite.sty is already installed on most LaTeX systems. Be sure and use
% version 5.0 (2009-03-20) and later if using hyperref.sty.
% The latest version can be obtained at:
% http://www.ctan.org/pkg/cite
% The documentation is contained in the cite.sty file itself.





\usepackage{graphicx}
% *** GRAPHICS RELATED PACKAGES ***
%
\ifCLASSINFOpdf
  % \usepackage[pdftex]{graphicx}
  % declare the path(s) where your graphic files are
  % \graphicspath{{../pdf/}{../jpeg/}}
  % and their extensions so you won't have to specify these with
  % every instance of \includegraphics
  % \DeclareGraphicsExtensions{.pdf,.jpeg,.png}
\else
  % or other class option (dvipsone, dvipdf, if not using dvips). graphicx
  % will default to the driver specified in the system graphics.cfg if no
  % driver is specified.
  % \usepackage[dvips]{graphicx}
  % declare the path(s) where your graphic files are
  % \graphicspath{{../eps/}}
  % and their extensions so you won't have to specify these with
  % every instance of \includegraphics
  % \DeclareGraphicsExtensions{.eps}
\fi
% graphicx was written by David Carlisle and Sebastian Rahtz. It is
% required if you want graphics, photos, etc. graphicx.sty is already
% installed on most LaTeX systems. The latest version and documentation
% can be obtained at: 
% http://www.ctan.org/pkg/graphicx
% Another good source of documentation is "Using Imported Graphics in
% LaTeX2e" by Keith Reckdahl which can be found at:
% http://www.ctan.org/pkg/epslatex
%
% latex, and pdflatex in dvi mode, support graphics in encapsulated
% postscript (.eps) format. pdflatex in pdf mode supports graphics
% in .pdf, .jpeg, .png and .mps (metapost) formats. Users should ensure
% that all non-photo figures use a vector format (.eps, .pdf, .mps) and
% not a bitmapped formats (.jpeg, .png). The IEEE frowns on bitmapped formats
% which can result in "jaggedy"/blurry rendering of lines and letters as
% well as large increases in file sizes.
%
% You can find documentation about the pdfTeX application at:
% http://www.tug.org/applications/pdftex





% *** MATH PACKAGES ***
%
\usepackage{amsmath}
% A popular package from the American Mathematical Society that provides
% many useful and powerful commands for dealing with mathematics.
% Do NOT use the amsbsy package under comsoc mode as that feature is
% already built into the Times Math font (newtxmath, mathtime, etc.).
% 
% Also, note that the amsmath package sets \interdisplaylinepenalty to 10000
% thus preventing page breaks from occurring within multiline equations. Use:
\interdisplaylinepenalty=2500
% after loading amsmath to restore such page breaks as IEEEtran.cls normally
% does. amsmath.sty is already installed on most LaTeX systems. The latest
% version and documentation can be obtained at:
% http://www.ctan.org/pkg/amsmath


% Select a Times math font under comsoc mode or else one will automatically
% be selected for you at the document start. This is required as Communications
% Society journals use a Times, not Computer Modern, math font.
\usepackage[cmintegrals]{newtxmath}
% The freely available newtxmath package was written by Michael Sharpe and
% provides a feature rich Times math font. The cmintegrals option, which is
% the default under IEEEtran, is needed to get the correct style integral
% symbols used in Communications Society journals. Version 1.451, July 28,
% 2015 or later is recommended. Also, do *not* load the newtxtext.sty package
% as doing so would alter the main text font.
% http://www.ctan.org/pkg/newtx
%
% Alternatively, you can use the MathTime commercial fonts if you have them
% installed on your system:
%\usepackage{mtpro2}
%\usepackage{mt11p}
%\usepackage{mathtime}


%\usepackage{bm}
% The bm.sty package was written by David Carlisle and Frank Mittelbach.
% This package provides a \bm{} to produce bold math symbols.
% http://www.ctan.org/pkg/bm





% *** SPECIALIZED LIST PACKAGES ***
%
%\usepackage{algorithmic}
% algorithmic.sty was written by Peter Williams and Rogerio Brito.
% This package provides an algorithmic environment fo describing algorithms.
% You can use the algorithmic environment in-text or within a figure
% environment to provide for a floating algorithm. Do NOT use the algorithm
% floating environment provided by algorithm.sty (by the same authors) or
% algorithm2e.sty (by Christophe Fiorio) as the IEEE does not use dedicated
% algorithm float types and packages that provide these will not provide
% correct IEEE style captions. The latest version and documentation of
% algorithmic.sty can be obtained at:
% http://www.ctan.org/pkg/algorithms
% Also of interest may be the (relatively newer and more customizable)
% algorithmicx.sty package by Szasz Janos:
% http://www.ctan.org/pkg/algorithmicx




% *** ALIGNMENT PACKAGES ***
%
%\usepackage{array}
% Frank Mittelbach's and David Carlisle's array.sty patches and improves
% the standard LaTeX2e array and tabular environments to provide better
% appearance and additional user controls. As the default LaTeX2e table
% generation code is lacking to the point of almost being broken with
% respect to the quality of the end results, all users are strongly
% advised to use an enhanced (at the very least that provided by array.sty)
% set of table tools. array.sty is already installed on most systems. The
% latest version and documentation can be obtained at:
% http://www.ctan.org/pkg/array


% IEEEtran contains the IEEEeqnarray family of commands that can be used to
% generate multiline equations as well as matrices, tables, etc., of high
% quality.




% *** SUBFIGURE PACKAGES ***
%\ifCLASSOPTIONcompsoc
%  \usepackage[caption=false,font=normalsize,labelfont=sf,textfont=sf]{subfig}
%\else
%  \usepackage[caption=false,font=footnotesize]{subfig}
%\fi
% subfig.sty, written by Steven Douglas Cochran, is the modern replacement
% for subfigure.sty, the latter of which is no longer maintained and is
% incompatible with some LaTeX packages including fixltx2e. However,
% subfig.sty requires and automatically loads Axel Sommerfeldt's caption.sty
% which will override IEEEtran.cls' handling of captions and this will result
% in non-IEEE style figure/table captions. To prevent this problem, be sure
% and invoke subfig.sty's "caption=false" package option (available since
% subfig.sty version 1.3, 2005/06/28) as this is will preserve IEEEtran.cls
% handling of captions.
% Note that the Computer Society format requires a larger sans serif font
% than the serif footnote size font used in traditional IEEE formatting
% and thus the need to invoke different subfig.sty package options depending
% on whether compsoc mode has been enabled.
%
% The latest version and documentation of subfig.sty can be obtained at:
% http://www.ctan.org/pkg/subfig




% *** FLOAT PACKAGES ***
%
%\usepackage{fixltx2e}
% fixltx2e, the successor to the earlier fix2col.sty, was written by
% Frank Mittelbach and David Carlisle. This package corrects a few problems
% in the LaTeX2e kernel, the most notable of which is that in current
% LaTeX2e releases, the ordering of single and double column floats is not
% guaranteed to be preserved. Thus, an unpatched LaTeX2e can allow a
% single column figure to be placed prior to an earlier double column
% figure.
% Be aware that LaTeX2e kernels dated 2015 and later have fixltx2e.sty's
% corrections already built into the system in which case a warning will
% be issued if an attempt is made to load fixltx2e.sty as it is no longer
% needed.
% The latest version and documentation can be found at:
% http://www.ctan.org/pkg/fixltx2e


%\usepackage{stfloats}
% stfloats.sty was written by Sigitas Tolusis. This package gives LaTeX2e
% the ability to do double column floats at the bottom of the page as well
% as the top. (e.g., "\begin{figure*}[!b]" is not normally possible in
% LaTeX2e). It also provides a command:
%\fnbelowfloat
% to enable the placement of footnotes below bottom floats (the standard
% LaTeX2e kernel puts them above bottom floats). This is an invasive package
% which rewrites many portions of the LaTeX2e float routines. It may not work
% with other packages that modify the LaTeX2e float routines. The latest
% version and documentation can be obtained at:
% http://www.ctan.org/pkg/stfloats
% Do not use the stfloats baselinefloat ability as the IEEE does not allow
% \baselineskip to stretch. Authors submitting work to the IEEE should note
% that the IEEE rarely uses double column equations and that authors should try
% to avoid such use. Do not be tempted to use the cuted.sty or midfloat.sty
% packages (also by Sigitas Tolusis) as the IEEE does not format its papers in
% such ways.
% Do not attempt to use stfloats with fixltx2e as they are incompatible.
% Instead, use Morten Hogholm'a dblfloatfix which combines the features
% of both fixltx2e and stfloats:
%
% \usepackage{dblfloatfix}
% The latest version can be found at:
% http://www.ctan.org/pkg/dblfloatfix




%\ifCLASSOPTIONcaptionsoff
%  \usepackage[nomarkers]{endfloat}
% \let\MYoriglatexcaption\caption
% \renewcommand{\caption}[2][\relax]{\MYoriglatexcaption[#2]{#2}}
%\fi
% endfloat.sty was written by James Darrell McCauley, Jeff Goldberg and 
% Axel Sommerfeldt. This package may be useful when used in conjunction with 
% IEEEtran.cls'  captionsoff option. Some IEEE journals/societies require that
% submissions have lists of figures/tables at the end of the paper and that
% figures/tables without any captions are placed on a page by themselves at
% the end of the document. If needed, the draftcls IEEEtran class option or
% \CLASSINPUTbaselinestretch interface can be used to increase the line
% spacing as well. Be sure and use the nomarkers option of endfloat to
% prevent endfloat from "marking" where the figures would have been placed
% in the text. The two hack lines of code above are a slight modification of
% that suggested by in the endfloat docs (section 8.4.1) to ensure that
% the full captions always appear in the list of figures/tables - even if
% the user used the short optional argument of \caption[]{}.
% IEEE papers do not typically make use of \caption[]'s optional argument,
% so this should not be an issue. A similar trick can be used to disable
% captions of packages such as subfig.sty that lack options to turn off
% the subcaptions:
% For subfig.sty:
% \let\MYorigsubfloat\subfloat
% \renewcommand{\subfloat}[2][\relax]{\MYorigsubfloat[]{#2}}
% However, the above trick will not work if both optional arguments of
% the \subfloat command are used. Furthermore, there needs to be a
% description of each subfigure *somewhere* and endfloat does not add
% subfigure captions to its list of figures. Thus, the best approach is to
% avoid the use of subfigure captions (many IEEE journals avoid them anyway)
% and instead reference/explain all the subfigures within the main caption.
% The latest version of endfloat.sty and its documentation can obtained at:
% http://www.ctan.org/pkg/endfloat
%
% The IEEEtran \ifCLASSOPTIONcaptionsoff conditional can also be used
% later in the document, say, to conditionally put the References on a 
% page by themselves.




% *** PDF, URL AND HYPERLINK PACKAGES ***
%
%\usepackage{url}
% url.sty was written by Donald Arseneau. It provides better support for
% handling and breaking URLs. url.sty is already installed on most LaTeX
% systems. The latest version and documentation can be obtained at:
% http://www.ctan.org/pkg/url
% Basically, \url{my_url_here}.




% *** Do not adjust lengths that control margins, column widths, etc. ***
% *** Do not use packages that alter fonts (such as pslatex).         ***
% There should be no need to do such things with IEEEtran.cls V1.6 and later.
% (Unless specifically asked to do so by the journal or conference you plan
% to submit to, of course. )


% correct bad hyphenation here
\hyphenation{op-tical net-works semi-conduc-tor}

\usepackage{array}
\usepackage{multirow}

\newcommand\MyBox[2]{
  \fbox{\lower0.75cm
    \vbox to 1.7cm{\vfil
      \hbox to 1.7cm{\hfil\parbox{1.4cm}{#1\\#2}\hfil}
      \vfil}%
  }%
}



\begin{document}
%
% paper title
% Titles are generally capitalized except for words such as a, an, and, as,
% at, but, by, for, in, nor, of, on, or, the, to and up, which are usually
% not capitalized unless they are the first or last word of the title.
% Linebreaks \\ can be used within to get better formatting as desired.
% Do not put math or special symbols in the title.
\title{Feature Selection\\ Fitness Landscape Analysis}
%
%
% author names and IEEE memberships
% note positions of commas and nonbreaking spaces ( ~ ) LaTeX will not break
% a structure at a ~ so this keeps an author's name from being broken across
% two lines.
% use \thanks{} to gain access to the first footnote area
% a separate \thanks must be used for each paragraph as LaTeX2e's \thanks
% was not built to handle multiple paragraphs
%

\author{Werner~Mostert}% <-this % stops a space

% note the % following the last \IEEEmembership and also \thanks - 
% these prevent an unwanted space from occurring between the last author name
% and the end of the author line. i.e., if you had this:
% 
% \author{....lastname \thanks{...} \thanks{...} }
%                     ^------------^------------^----Do not want these spaces!
%
% a space would be appended to the last name and could cause every name on that
% line to be shifted left slightly. This is one of those "LaTeX things". For
% instance, "\textbf{A} \textbf{B}" will typeset as "A B" not "AB". To get
% "AB" then you have to do: "\textbf{A}\textbf{B}"
% \thanks is no different in this regard, so shield the last } of each \thanks
% that ends a line with a % and do not let a space in before the next \thanks.
% Spaces after \IEEEmembership other than the last one are OK (and needed) as
% you are supposed to have spaces between the names. For what it is worth,
% this is a minor point as most people would not even notice if the said evil
% space somehow managed to creep in.



% The paper headers
\markboth{COS 700 Copy}%
{Shell \MakeLowercase{\textit{et al.}}: Bare Demo of IEEEtran.cls for IEEE Communications Society Journals}
% The only time the second header will appear is for the odd numbered pages
% after the title page when using the twoside option.
% 
% *** Note that you probably will NOT want to include the author's ***
% *** name in the headers of peer review papers.                   ***
% You can use \ifCLASSOPTIONpeerreview for conditional compilation here if
% you desire.


% If you want to put a publisher's ID mark on the page you can do it like
% this:
%\IEEEpubid{0000--0000/00\$00.00~\copyright~2015 IEEE}
% Remember, if you use this you must call \IEEEpubidadjcol in the second
% column for its text to clear the IEEEpubid mark.

% use for special paper notices
%\IEEEspecialpapernotice{(Invited Paper)}

% make the title area
\maketitle

% As a general rule, do not put math, special symbols or citations
% in the abstract or keywords.
\begin{abstract}
Feature selection is a complex problem which has been addressed in many different ways. Feature selection algorithms that have been developed are often computationally expensive; provide
an insignificant increase in predictor performance and can even start overfitting. The proposed
research will investigate the combinatorial problem of selecting feature sets for optimal performance by a given classifier. By analysing the fitness landscape of the feature selection problem
it is envisaged to better understand the deeper problem. Landscape characteristics that have an influence on the search algorithms are investigated. It is hoped that the analysis
provides insight into choosing a warranted feature selection algorithm for future research.
\end{abstract}

% Note that keywords are not normally used for peerreview papers.
\begin{IEEEkeywords}
Feature Selection, Fitness Landscape, Landscape Analysis
\end{IEEEkeywords}

% For peer review papers, you can put extra information on the cover
% page as needed:
% \ifCLASSOPTIONpeerreview
% \begin{center} \bfseries EDICS Category: 3-BBND \end{center}
% \fi
%
% For peerreview papers, this IEEEtran command inserts a page break and
% creates the second title. It will be ignored for other modes.
\IEEEpeerreviewmaketitle



\section{Introduction}

It is a known and easily comprehensible fact that humans effortlessly find patterns in their daily lives. The recognition of these patterns allows for the perception of arbitrary objects such as other humans, physical objects etc. Within the field of Machine Learning it often times attempted to accomplish these pattern recognition goals \cite{simon2013too} since it is of such great value in fields such as Computer Vision  \cite{wernick2010machine}.\\
Various techniques to date have been developed in the form of classification techniques \cite{nguyen2008survey} , otherwise referred to as classifiers, in order to recognize patterns in data. Classifiers generally work on a similar premise: some conclusion is made with respect to a set of parameters in data. These parameters in the observed data can also be called features. Chandrashekar and Sahin define a feature as "an individual measurable property of the process being observed" \cite{chandrashekar2014survey}. Intuitively speaking, when a human is given the task of recognizing a friend or family member, how can the individual be identified, i.e. classified in terms of identity? The human may perhaps take into account physical features such as the person's facial features, voice or even smell. A classifier would similarly take into account some set or subset of all possible features that could conceptually be present in this scenario in order to classify the person as a unique individual. Given the fact that the classifier uses features to do classification, it becomes a complex problem to decide on which features are the most information rich to provide for an accurate classification.\\
The problem of feature selection is that of selecting a subset of all available features. Feature selection is beneficial in order to better understand and visualize data as well as improving classification performance by effectively reducing the dimensionality of the problem \cite{guyon2003introduction}. Modern problems such as image classification suffer from extremely high dimensionality, thus affecting classifier performance. There exist many approaches of feature selection of which many have proven to be problem dependent and have had varying measures of success \cite{chandrashekar2014survey}. The issue of feature irrelevance comes to light in terms of feature selection for classifiers. Two features that are considered within mutual exclusion could be useless, but the union of these features could be of massive importance \cite{guyon2003introduction}. A primitive approach to solving this problem would be to do an exhaustive search of the combination of features to use that provides optimal performance. Given a small number of features this is conceptually possible, however, 
 as stated by Amaldi  et. al \cite{amaldi1998approximability} the full permutation of feature sets for a highly dimensional problem is a Non-Polynomial (NP)-hard problem.\\
In order to further the development of a generalized theoretical framework for feature selection and to better understand the problem of feature selection this paper analyses the fitness landscape of the feature selection space with respect to simple and non-stochastic classifier.

\section{Background}

There are various feature selection algorithms which have been developed to date, and there is still current research on the topic \cite{kira1992feature, jain1997feature, kohavi1997wrappers, yu2003feature, yeh2016feature, ge2016mctwo}. These generally fall into three categories namely Filter Methods, Wrapper Methods and Embedded Methods \cite{chandrashekar2014survey}. There is evidently no shortage of algorithms for conducting feature selection. The algorithms are extensively diverse in how the problem is approached. There is however no theoretical framework in place to guide researchers in making decisions on which algorithms to use \cite{guyon2003introduction}. As stated byGuyon and Elliseeff, feature selection methods may be used to improve classifier performance. They did however find that for problems of high dimensionality the performance increase is not always significant.  As stated by Pitzer et al. "the most common motivation for fitness landscape analysis is to gain a better understanding of algorithm performance on a related set of problem instances." \cite{pitzer2012comprehensive}. Since the feature selection problem is rather complex it is useful to do landscape analysis in order to be able to obtain more detailed understanding of the whole problem. An underlying fitness landscape can be a valuable tool for analysis for heuristic search algorithms \cite{pitzer2012comprehensive}. There are various characteristics of a fitness landscape that can be investigated and also many different techniques that may be applied in order to analyze these characteristics \cite{malan2013survey} . Some fitness landscapes possess structural attributes that can lead search algorithms to good or bad solutions; these measurements are normally only indicative for a certain algorithm \cite{malan2013survey}.  

NOTE: Not done yet - still need to allow for a better explanation of fitness landscape analysis

\section{Implementation}
\subsection{Representation}

Since the problem of choosing a subset of features from the complete feature set $F$ is a combinatorial problem, the subset of selected features $F_s$ can be represented as a binary string. An 'on' bit in the string represents the inclusion of the feature, whereas conversely, an 'off' represents its exclusion. In this format it is possible to model a full permutation of all possible inclusions and exclusions of subsets, of the complete feature set. In order to analyze a fitness landscape a precondition is that it is computationally possible to calculate a fitness value at a given point in the landscape (i.e. a specific configuration of 'on' and 'off' in the binary string). Using a binary string with a valid fitness function one may construct a binary landscape which can be analyzed.

\subsection{Fitness Function}
Since one of the goals of feature selection is to be able to reduce the dimensionality of the classifier input, thus transitively improving performance, a measure of classification accuracy is used as the fitness value at a point. The choice of fitness function is of paramount importance when conducting fitness landscape analysis since there exists a variant of the landscape for disparate fitness functions and for variations in calculating the fitness based on different notions of neighborhood for the same fitness function \cite{malan2013survey}. It is desirable to be able to reliably reproduce the same landscape for a static data set. Fitness functions that are stochastic in nature such as Artificial Neural Networks using Stochastic Gradient Descent or Random Decision Forests could very likely provide for valid classification accuracy measurements but would introduce noise into the fitness landscape. This is due to the fact that since a stochastic classifier is generally executed multiple times and the resultant mean performance measurement is used. In such a case by reproducing the fitness landscape the difference between resultant curves within the hyper-dimensional problem space would become fuzzy due to stochasticity. Therefor the classic $k$-nearest-neighbor \cite{cunningham2007k}, a simple non-stochastic classifier is used. The $k$-nearest neighbor is essentially a clustering classifier which works primarily on the concept of a neighborhood. In the case of this binary representations, the neighborhood can be calculated as a $k$ number of instances or points that are the closest to the sampled point in terms of the hamming distance of a point with respect to the sampled point. The hamming distance essentially calculates the difference (or similarity) between two bit strings, therefor the most similar points in the problem space are considered the neighbors of the sampled point.

\begin{figure}[H]
\centering
\includegraphics[width=0.5\linewidth]{kNearestNeighbour.png}
\caption{$k$ - Nearest Neighbor}
\end{figure}

In order to the obtain a fitness value Cohen's Kappa-statistic , a non-biased, measure of classification accuracy is used. Cohen's Kappa is defined as \cite{ben2008comparison}:

\begin{equation}
	K = \frac{P_0 - P_c}{1 - P_c}
\end{equation}

where $P_c$ is the agreement probability as a result of randomness and $P_0$ is the total agreement probability. The concept of agreement probability is quite simple. Given a confusion matrix as in figure \ref{fig:confuse}, the Kappa statistic would be calculated as follows:


$$
P_c = (\frac{86}{100})(\frac{84}{100}) + (\frac{14}{100})(\frac{16}{100}) = 0.0162
$$
$$
P_0 = \frac{75}{100} + \frac{5}{100} = 0.8
$$
\begin{equation}
K = \frac{0.8 - 0.0162}{1 - 0.0162} = 0.7967
\end{equation}
\hfill\\

The rate of agreement in this case is therefore 0.7967. The Kappa statistic ranges from total disagreement at -1 through completely random classification to 1 which indicates total agreement. The Kappa statistic allows for the level of agreement for each class label to be measured. This is important since for a raw count of correct classification, the results may be statistically biased due to an overwhelming presence or absence of a specific class in an observed data set.

\begin{figure}[H]
\centering
\begin{tabular}{ r|c|c|l }
\multicolumn{1}{r}{}
 &  \multicolumn{1}{c}{A}
 & \multicolumn{1}{c}{B}
& \multicolumn{1}{l}{Total} \\
\cline{2-3}
A & 75 & 11 & 86 \\
\cline{2-3}
B & 9 & 5 & 14 \\
\cline{2-3}
\multicolumn{1}{r}{Total}
 &  \multicolumn{1}{c}{84}
 & \multicolumn{1}{c}{16}
& \multicolumn{1}{l}{100}

\end{tabular}
\hfill\\\hfill\\
\caption{Confusion Matrix 1 for 100 samples}
\label{fig:confuse}
\end{figure}

\subsection{Data Sets}

A variety of data are utilized in order to conduct analysis. A total of X(to be filled in) data sets are used, which contains nominal and numerical data elements or features. No specific preprocessing of the data sets take place. The UCI repository is used to source a variety of data sets. Sample data sets range between X and X output classes with varying degrees of dimensionality from X(to be filled in) to X(to be filled in) features.

\subsection{Random Walk}

One of the core practices of landscape analysis is the concept of sampling the problem space in order to gain sufficient information about the problem space. There are various ways to accomplish this task. One such method is uniform sampling. Within the context of binary landscapes this entails choosing a random bit string which maps to a configuration within the hypercube of all possible configurations. Naively one may convert the bit string of length $N$ to a real value with the formula $x = 2^N-1$ and sample a value from a uniform distribution constrained by 0 and $x$. However, for a large value of $N$ (as is the case for hyper-dimensional problems) this number becomes exponentially large and difficult to deal with. Another possible solution is to decide for each value within the bit string its on or off value based on if its over or under the median of any constrained uniform distribution, such as 0.5 between 0 and 1. A uniform sample though presents an issue. Problems that are of vastly high dimensionality have an excessively large search space and simple sample with an insufficient number of points sampled with respect to the size of the search space may heed misleading results.\\

NOTE: Discuss with Dr. Malan the difference between uniform sampling from a highly dimensional problem space vs doing a random walk on an isometric landscape. Also, is it a guarantee that binary landscapes are isometric?

\subsection{Tools and Experimental Procedure}

NOTE: Explain that weka was used as a data mining utility and remember to reference it

\subsection{Assumptions and Limitations}

NOTE: Explain here that a single classifier is used and that the landscapes would probably differ between classifiers for the same data sets. It may be useful in future for research to see what the difference is between the landscapes that are analyzed when conducting the analysis on the same data sets for different classifiers. 

\section{Fitness Landscape Characteristics}
\subsection{Fitness Distribution}

NOTE: Explain here what fitness distribution is, and how it is interpreted

\subsection{Hamming Distance between Levels}

NOTE: Explain here what hamming distance between levels is and how to interpret it

\subsection{Some other characteristic possibly}

NOTE: to be confirmed

\section{Research Results}
\subsection{Fitness Distribution}
\subsection{Hamming Distance between Levels}


% An example of a floating figure using the graphicx package.
% Note that \label must occur AFTER (or within) \caption.
% For figures, \caption should occur after the \includegraphics.
% Note that IEEEtran v1.7 and later has special internal code that
% is designed to preserve the operation of \label within \caption
% even when the captionsoff option is in effect. However, because
% of issues like this, it may be the safest practice to put all your
% \label just after \caption rather than within \caption{}.
%
% Reminder: the "draftcls" or "draftclsnofoot", not "draft", class
% option should be used if it is desired that the figures are to be
% displayed while in draft mode.
%
%\begin{figure}[!t]
%\centering
%\includegraphics[width=2.5in]{myfigure}
% where an .eps filename suffix will be assumed under latex, 
% and a .pdf suffix will be assumed for pdflatex; or what has been declared
% via \DeclareGraphicsExtensions.
%\caption{Simulation results for the network.}
%\label{fig_sim}
%\end{figure}

% Note that the IEEE typically puts floats only at the top, even when this
% results in a large percentage of a column being occupied by floats.


% An example of a double column floating figure using two subfigures.
% (The subfig.sty package must be loaded for this to work.)
% The subfigure \label commands are set within each subfloat command,
% and the \label for the overall figure must come after \caption.
% \hfil is used as a separator to get equal spacing.
% Watch out that the combined width of all the subfigures on a 
% line do not exceed the text width or a line break will occur.
%
%\begin{figure*}[!t]
%\centering
%\subfloat[Case I]{\includegraphics[width=2.5in]{box}%
%\label{fig_first_case}}
%\hfil
%\subfloat[Case II]{\includegraphics[width=2.5in]{box}%
%\label{fig_second_case}}
%\caption{Simulation results for the network.}
%\label{fig_sim}
%\end{figure*}
%
% Note that often IEEE papers with subfigures do not employ subfigure
% captions (using the optional argument to \subfloat[]), but instead will
% reference/describe all of them (a), (b), etc., within the main caption.
% Be aware that for subfig.sty to generate the (a), (b), etc., subfigure
% labels, the optional argument to \subfloat must be present. If a
% subcaption is not desired, just leave its contents blank,
% e.g., \subfloat[].


% An example of a floating table. Note that, for IEEE style tables, the
% \caption command should come BEFORE the table and, given that table
% captions serve much like titles, are usually capitalized except for words
% such as a, an, and, as, at, but, by, for, in, nor, of, on, or, the, to
% and up, which are usually not capitalized unless they are the first or
% last word of the caption. Table text will default to \footnotesize as
% the IEEE normally uses this smaller font for tables.
% The \label must come after \caption as always.
%
%\begin{table}[!t]
%% increase table row spacing, adjust to taste
%\renewcommand{\arraystretch}{1.3}
% if using array.sty, it might be a good idea to tweak the value of
% \extrarowheight as needed to properly center the text within the cells
%\caption{An Example of a Table}
%\label{table_example}
%\centering
%% Some packages, such as MDW tools, offer better commands for making tables
%% than the plain LaTeX2e tabular which is used here.
%\begin{tabular}{|c||c|}
%\hline
%One & Two\\
%\hline
%Three & Four\\
%\hline
%\end{tabular}
%\end{table}


% Note that the IEEE does not put floats in the very first column
% - or typically anywhere on the first page for that matter. Also,
% in-text middle ("here") positioning is typically not used, but it
% is allowed and encouraged for Computer Society conferences (but
% not Computer Society journals). Most IEEE journals/conferences use
% top floats exclusively. 
% Note that, LaTeX2e, unlike IEEE journals/conferences, places
% footnotes above bottom floats. This can be corrected via the
% \fnbelowfloat command of the stfloats package.


\section{Conclusion}

NOTE: To be completed





% if have a single appendix:
%\appendix[Proof of the Zonklar Equations]
% or
%\appendix  % for no appendix heading
% do not use \section anymore after \appendix, only \section*
% is possibly needed

% use appendices with more than one appendix
% then use \section to start each appendix
% you must declare a \section before using any
% \subsection or using \label (\appendices by itself
% starts a section numbered zero.)
%


\appendices
\section{Proof of the First Zonklar Equation}
Appendix one text goes here.


The authors would like to thank...


% Can use something like this to put references on a page
% by themselves when using endfloat and the captionsoff option.
\ifCLASSOPTIONcaptionsoff
  \newpage
\fi



% trigger a \newpage just before the given reference
% number - used to balance the columns on the last page
% adjust value as needed - may need to be readjusted if
% the document is modified later
%\IEEEtriggeratref{8}
% The "triggered" command can be changed if desired:
%\IEEEtriggercmd{\enlargethispage{-5in}}

% references section

% can use a bibliography generated by BibTeX as a .bbl file
% BibTeX documentation can be easily obtained at:
% http://mirror.ctan.org/biblio/bibtex/contrib/doc/
% The IEEEtran BibTeX style support page is at:
% http://www.michaelshell.org/tex/ieeetran/bibtex/
%\bibliographystyle{IEEEtran}
% argument is your BibTeX string definitions and bibliography database(s)
%\bibliography{IEEEabrv,../bib/paper}
%


\bibliographystyle{IEEEtran} 
\bibliography{references}
% that's all folks
\end{document}


